\documentclass[a4j]{jsarticle}
\setlength{\topmargin}{-20.4cm}
\setlength{\oddsidemargin}{-10.4mm}
\setlength{\evensidemargin}{-10.4mm}
\setlength{\textwidth}{18cm}
\setlength{\textheight}{26cm}

\usepackage[top=15truemm,bottom=20truemm,left=20truemm,right=20truemm]{geometry}
\usepackage[latin1]{inputenc}
\usepackage{amsmath}
\usepackage{amsfonts}
\usepackage{amssymb}
\usepackage[dvipdfmx]{graphicx}
\usepackage[hang,small,bf]{caption}
\usepackage[subrefformat=parens]{subcaption}
\usepackage[dvipdfmx]{color}
\usepackage{listings}
\usepackage{listings,jvlisting}
\usepackage{geometry}
\usepackage{framed}
\usepackage{color}
\usepackage[dvipdfmx]{hyperref}
\usepackage{ascmac}
\usepackage{enumerate}
\usepackage{tabularx}
\usepackage{cancel}
\usepackage{scalefnt}
\usepackage{overcite}
\usepackage{otf}
\usepackage{multicol}
\usepackage[geometry]{ifsym}

\renewcommand{\figurename}{Fig.}
\renewcommand{\tablename}{Table }

\lstset{
basicstyle={\ttfamily},
identifierstyle={\small},
commentstyle={\smallitshape},
keywordstyle={\small\bfseries},
ndkeywordstyle={\small},
stringstyle={\small\ttfamily},
frame={tb},
breaklines=true,
columns=[l]{fullflexible},
xrightmargin=0zw,
xleftmargin=3zw,
numberstyle={\scriptsize},
stepnumber=1,
numbersep=1zw,
lineskip=-0.5ex
}

% キャプション後ろのダブルコロンを消す
\makeatletter
\long\def\@makecaption#1#2{%
  \vskip\abovecaptionskip
  \iftdir\sbox\@tempboxa{#1\hskip1zw#2}%
    \else\sbox\@tempboxa{#1 #2}%
  \fi
  \ifdim \wd\@tempboxa >\hsize
    \iftdir #1\hskip1zw#2\relax\par
      \else #1 #2\relax\par\fi
  \else
    \global \@minipagefalse
    \hbox to\hsize{\hfil\box\@tempboxa\hfil}%
  \fi
  \vskip\belowcaptionskip}
\makeatother

% タイトル
\makeatletter
\def\@maketitle
{
\begin{center}
{\LARGE \@title \par}
\end{center}
\begin{flushright}
{\large \@date 報告書 No.24}\\
{\large M2 \@author}
\end{flushright}
\par\vskip 1.5em
}
\makeatother

\author{来代 勝胤}
\title{令和4年度 4月 第1週 報告書}
\date{2022/4/4}

\begin{document}
\maketitle

\section*{報告内容}
\begin{enumerate}[1.]
  \item 計測アルゴリズムの精度評価
  \item 今後の予定
\end{enumerate}

\section*{進捗状況}
ケーシングなし・回転ありのタイヤモデルの後流について撮影を行った.
その結果,対象物の後流では流れが減速するため,
対応したPTVアルゴリズムが必要であるとわかった.

\section{計測アルゴリズムの精度評価}
\subsection{データセット作成}
\subsection{シミュレーション条件}
\begin{table}[hbtp]
  \label{table:data_type}
  \caption{シミュレーション条件}
  \centering
  \begin{tabular}{ c | c | c | c }
           & \textgt{角速度} [deg/s] & \textgt{粒子数密度} [$-/\mathrm{mm}^3$] & \textgt{精度の予測} \\ \hline
    Case 1 & 5.0                     &                                         & \Circle             \\ \hline
    Case 2 & 5.0                     &                                         & \Circle             \\ \hline
    Case 3 & 5.0                     &                                         & \TriangleUp         \\ \hline
    Case 4 & 10.0                    &                                         & \Circle             \\ \hline
    Case 5 & 10.0                    &                                         & \TriangleUp         \\ \hline
    Case 6 & 10.0                    &                                         & \Cross              \\ \hline
    Case 7 & 15.0                    &                                         & \TriangleUp         \\ \hline
    Case 8 & 15.0                    &                                         & \Cross              \\ \hline
    Case 9 & 15.0                    &                                         & \Cross              \\ \hline
  \end{tabular}
\end{table}


\section{今後の予定}
\begin{itemize}
  \item 粒子に対応したPTVプログラムの作成
\end{itemize}

\end{document}