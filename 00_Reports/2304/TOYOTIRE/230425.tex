\documentclass[twocolumn,a4j]{jsarticle}
\bibliographystyle{junsrt}

\setlength{\topmargin}{-20.4cm}
\setlength{\oddsidemargin}{-10.4mm}
\setlength{\evensidemargin}{-10.4mm}
\setlength{\textwidth}{18cm}
\setlength{\textheight}{26cm}

\usepackage[top=15truemm,bottom=20truemm,left=20truemm,right=20truemm]{geometry}
\usepackage[latin1]{inputenc}
\usepackage{amsmath}
\usepackage{amsfonts}
\usepackage{amssymb}
\usepackage[dvipdfmx]{graphicx}
\usepackage[hang,small,bf]{caption}
\usepackage[subrefformat=parens]{subcaption}
\usepackage[dvipdfmx]{color}
\usepackage{listings}
\usepackage{listings,jvlisting}
\usepackage{geometry}
\usepackage{framed}
\usepackage{color}
\usepackage[dvipdfmx]{hyperref}
\usepackage{ascmac}
\usepackage{enumerate}
\usepackage{tabularx}
\usepackage{cancel}
\usepackage{scalefnt}
\usepackage{overcite}
\usepackage{otf}

\renewcommand{\figurename}{Fig.}
\renewcommand{\tablename}{Table }

\hypersetup{%
    hidelinks %リンクの色消し
}

\lstset{
basicstyle={\ttfamily},
identifierstyle={\small},
commentstyle={\smallitshape},
keywordstyle={\small\bfseries},
ndkeywordstyle={\small},
stringstyle={\small\ttfamily},
frame={tb},
breaklines=true,
columns=[l]{fullflexible},
xrightmargin=0zw,
xleftmargin=3zw,
numberstyle={\scriptsize},
stepnumber=1,
numbersep=1zw,
lineskip=-0.5ex
}

% キャプション後ろのダブルコロンを消す
\makeatletter
\long\def\@makecaption#1#2{%
  \vskip\abovecaptionskip
  \iftdir\sbox\@tempboxa{#1\hskip1zw#2}%
    \else\sbox\@tempboxa{#1 #2}%
  \fi
  \ifdim \wd\@tempboxa >\hsize
    \iftdir #1\hskip1zw#2\relax\par
      \else #1 #2\relax\par\fi
  \else
    \global \@minipagefalse
    \hbox to\hsize{\hfil\box\@tempboxa\hfil}%
  \fi
  \vskip\belowcaptionskip}
\makeatother


\makeatletter
\def\@maketitle
{
\begin{center}
{\LARGE \@title \par}
\end{center}
\begin{flushright}
{\large \@date}\\
{\large 京都工芸繊維大学 大学院 機械設計学専攻 計測システム工学研究室}\\
{\large M2 \@author}
\end{flushright}
\par\vskip 1.5em
}
\makeatother

\author{来代 勝胤 / KITADAI Masatsugu}
\title{令和5年度 4月度 共同研究 報告書}
\date{2023/04/25}

\begin{document}
\columnseprule=0.1mm
\maketitle

\section*{報告内容}
\begin{enumerate}[1.]
    \item 数値シミュレーション
    \item 真値の作成
    \item 計測精度の誤差評価
    \item 研究発表について
    \item 5月の予定
\end{enumerate}
\section*{進捗状況}

\section{数値シミュレーション}

\subsection{シミュレーション条件}
\begin{table}[hbtp]
    \label{table:data_type}
    \caption{シミュレーション条件}
    \centering
    \begin{tabular}{ c c | r l}
        \hline
        粒子数密度              & $n$          & 170                    & [個/枚]            \\ \hline
        壁の回転速度            & $\omega$     & 10                     & [deg/s]            \\ \hline
        動粘性係数              & $\nu$        & $1.004 \times 10^{-6}$ & [$\mathrm{m}^2$/s] \\ \hline
        $\mathrm{LLS}_1$ の位置 & $x_0$        & 7.000                  & [mm]               \\ \hline
        $\mathrm{LLS}_1$ の厚み & $T_1$        & $3.086\times 10^{-3}$  & [mm]               \\ \hline
        $\mathrm{LLS}_2$ の厚み & $T_2$        & $9.259\times 10^{-3}$  & [mm/s]             \\ \hline
        LLS 間の距離            & $\Delta x$   & $9.645\times 10^{-3}$  & [mm/s]             \\ \hline
        撮影範囲                & $y \times z$ & $40 \times 40$         & [mm]               \\ \hline
        画像サイズ              & $w \times h$ & $800 \times 800$       & [px]               \\ \hline
    \end{tabular}
\end{table}

\begin{table}[hbtp]
    \label{table:data_type}
    \caption{実験条件(参考)}
    \centering
    \begin{tabular}{ c c | r l}
        \hline
        % 主流速度       & $u$          & 250              & [mm/s] \\ \hline
        粒子数密度              & $n$            & 70                     & [個/枚]            \\ \hline
        壁の回転速度            & $\omega'$      & -                      & [deg/s]            \\ \hline
        動粘性係数              & $\nu'$         & $1.004 \times 10^{-6}$ & [$\mathrm{m}^2$/s] \\ \hline
        $\mathrm{LLS}_1$ の位置 & $x'_0$         & -                      & [mm]               \\ \hline
        $\mathrm{LLS}_1$ の厚み & $T'_1$         & 1.000                  & [mm]               \\ \hline
        $\mathrm{LLS}_2$ の厚み & $T'_2$         & 3.000                  & [mm/s]             \\ \hline
        LLS 間の距離            & $\Delta x'$    & 3.125                  & [mm/s]             \\ \hline
        撮影範囲                & $y' \times z'$ & $100 \times 50$        & [mm]               \\ \hline
        画像サイズ              & $w' \times h'$ & $800 \times 400$       & [px]               \\ \hline
    \end{tabular}
\end{table}


\section{真値の作成}


\section{計測精度の誤差評価}


\section{研究発表について}

\subsection{第51回 可視化情報シンポジウム}
\subsection{ISTP-33}
\subsection{日本実験力学会 2023年度年次講演会}


\section{5月の予定}
\begin{itemize}
    \item
    \item 車両モデル周りの流れ場計測
    \item ISTP 論文提出 (5/30)
    \item 可視化情報シンポジウム 原稿提出 (5/30)
\end{itemize}


\end{document}