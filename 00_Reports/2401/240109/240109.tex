% !TeX program = platex_custom
\documentclass[twocolumn,a4j]{jsarticle}
\usepackage[top=15truemm,bottom=20truemm,left=20truemm,right=20truemm]{geometry}
\usepackage{amsmath}
\usepackage{amsfonts}
\usepackage{amssymb}
\usepackage[dvipdfmx]{graphicx}
\usepackage[hang,small,bf]{caption}
\usepackage[subrefformat=parens]{subcaption}
\usepackage[dvipdfmx]{color}
\usepackage{listings}
\usepackage{listings,jvlisting}
\usepackage{framed}
\usepackage[dvipdfmx]{hyperref}
\usepackage{ascmac}
\usepackage{enumerate}
\usepackage{tabularx}
\usepackage{cancel}
\usepackage{scalefnt}
\usepackage{overcite}
\usepackage{otf}
\usepackage{multicol}
\usepackage[geometry]{ifsym}
\usepackage{array}

\renewcommand{\figurename}{Fig.}
\renewcommand{\tablename}{Table }

\lstset{
basicstyle={\ttfamily},
identifierstyle={\small},
commentstyle={\smallitshape},
keywordstyle={\small\bfseries},
ndkeywordstyle={\small},
stringstyle={\small\ttfamily},
frame={tb},
breaklines=true,
columns=[l]{fullflexible},
xrightmargin=0zw,
xleftmargin=3zw,
numberstyle={\scriptsize},
stepnumber=1,
numbersep=1zw,
lineskip=-0.5ex
}

% キャプション後ろのダブルコロンを消す
\makeatletter
\long\def\@makecaption#1#2{%
  \vskip\abovecaptionskip
  \iftdir\sbox\@tempboxa{#1\hskip1zw#2}%
    \else\sbox\@tempboxa{#1 #2}%
  \fi
  \ifdim \wd\@tempboxa >\hsize
    \iftdir #1\hskip1zw#2\relax\par
      \else #1 #2\relax\par\fi
  \else
    \global \@minipagefalse
    \hbox to\hsize{\hfil\box\@tempboxa\hfil}%
  \fi
  \vskip\belowcaptionskip}
\makeatother

% タイトル
\makeatletter
\def\@maketitle
{
\begin{center}
{\LARGE \@title \par}
\end{center}
\begin{flushright}
{\large \@date 報告書 No.41}\\
{\large M2 \@author}
\end{flushright}
\par\vskip 1.5em
}
\makeatother

\author{来代 勝胤}
\title{令和5年度 1月 第2週 報告書}
\date{2024/1/9}

\begin{document}
\columnseprule=0.1mm
\maketitle

\section*{修士論文 目次}
\section{緒言}
\subsection{研究背景}
\subsection{研究目的}

\section{計測法}
\subsection{PIV計測手法}
\vskip 0.5 \baselineskip
\subsection{粒子クラスタ追跡法}
\subsubsection{PIVによる二次流れ計測}
\subsubsection{粒子像のクラスタリング}
\subsubsection{クラスタマッチング}
\subsubsection{粒子クラスタ中心}
\subsubsection{移動量解析}
\subsubsection{速度推定}

\vskip \baselineskip
\section{数値シミュレーション}
\subsection{撮影シミュレーションモデル}
\subsubsection{水槽座標系}
\subsubsection{カメラ座標系}
\subsubsection{画像座標系}
\subsubsection{画像生成}

\vskip 0.5 \baselineskip
\subsection{空間校正}
\subsubsection{校正画像の作成}
\subsubsection{校正性能評価}

\vskip 0.5 \baselineskip
\subsection{一様流}
\subsubsection{粒子像の生成}
\subsubsection{粒子像の識別}
\subsubsection{粒子位置検出}
\subsubsection{解析結果}
\subsubsection{性能評価}

\vskip 0.5 \baselineskip
\subsection{三角翼まわりの流れ}
\subsubsection{数値解析による流れ場の取得}
\subsubsection{解析モデル}
\subsubsection{粒子像の生成}
\subsubsection{解析結果}
\subsubsection{性能評価}

\newpage
\section{基本実験}
\vskip 0.5 \baselineskip
\subsection{実験装置}
\subsubsection{回流水槽}
\subsubsection{校正ブロック}

\vskip 0.5 \baselineskip
\subsection{一様流の撮影}
\subsubsection{撮影結果}

\subsection{三角翼後流の撮影}
\subsubsection{三角翼モデル}
\subsubsection{撮影結果}

\vskip \baselineskip
\section{応用測定実験}
\subsection{実験装置}
\subsubsection{車両モデル}
\subsubsection{撮影結果}

\vskip \baselineskip
\section{結言}

\end{document}