\documentclass[twocolumn,a4j]{jsarticle}
\setlength{\topmargin}{-18.5cm}
\setlength{\oddsidemargin}{-8.5mm}
\setlength{\evensidemargin}{-8.5mm}
\setlength{\textwidth}{19cm}
\setlength{\textheight}{26.5cm}

\usepackage[top=15truemm,bottom=20truemm,left=20truemm,right=20truemm]{geometry}
\usepackage[latin1]{inputenc}
\usepackage{amsmath}
\usepackage{amsfonts}
\usepackage{amssymb}
\usepackage[dvipdfmx]{graphicx}
\usepackage[hang,small,bf]{caption}
\usepackage[subrefformat=parens]{subcaption}
\usepackage[dvipdfmx]{color}
\usepackage{listings}
\usepackage{listings,jvlisting}
\usepackage{geometry}
\usepackage{framed}
\usepackage{color}
\usepackage[dvipdfmx]{hyperref}
\usepackage{ascmac}
\usepackage{enumerate}
\usepackage{tabularx}
\usepackage{cancel}
\usepackage{scalefnt}
\usepackage{overcite}
\usepackage{otf}
\usepackage{multicol}
\usepackage[geometry]{ifsym}

\renewcommand{\figurename}{Fig.}
\renewcommand{\tablename}{Table }

\hypersetup{%
    hidelinks %リンクの色消し
}

\lstset{
basicstyle={\ttfamily},
identifierstyle={\small},
commentstyle={\smallitshape},
keywordstyle={\small\bfseries},
ndkeywordstyle={\small},
stringstyle={\small\ttfamily},
frame={tb},
breaklines=true,
columns=[l]{fullflexible},
xrightmargin=0zw,
xleftmargin=3zw,
numberstyle={\scriptsize},
stepnumber=1,
numbersep=1zw,
lineskip=-0.5ex
}

% キャプション後ろのダブルコロンを消す
\makeatletter
\long\def\@makecaption#1#2{%
  \vskip\abovecaptionskip
  \iftdir\sbox\@tempboxa{#1\hskip1zw#2}%
    \else\sbox\@tempboxa{#1 #2}%
  \fi
  \ifdim \wd\@tempboxa >\hsize
    \iftdir #1\hskip1zw#2\relax\par
      \else #1 #2\relax\par\fi
  \else
    \global \@minipagefalse
    \hbox to\hsize{\hfil\box\@tempboxa\hfil}%
  \fi
  \vskip\belowcaptionskip}
\makeatother


\makeatletter
\def\@maketitle
{
\begin{center}
{\LARGE \@title \par}
\end{center}
\begin{flushright}
{\large \@date}\\
{\large 京都工芸繊維大学 大学院 機械設計学専攻 計測システム工学研究室}\\
{\large M2 \@author}
\end{flushright}
\par\vskip 1.5em
}
\makeatother

\author{来代 勝胤 / KITADAI Masatsugu}
\title{令和5年度 11月度 月例報告書}
\date{2023/12/05}

\begin{document}
\columnseprule=0.1mm
\maketitle

\section*{報告内容}
\begin{enumerate}[1.]
  \item [0.] 修士論文について
  \item 序論
  \item 粒子クラスタを用いた粒子追跡アルゴリズム
  \item 11月の予定
\end{enumerate}

\section*{進捗報告}

\setcounter{section}{-1}
\section{修士論文について}

\begin{table}[hbtp]
  \section*{$\blacksquare$ 題目}
  \centering
  \begin{tabular}{c}
    \hline
    多重カラーLLSを用いた二次流れPIV計測法  \\ \hline
    PIV Mesurement Method of Secondary Flow \\
    Using Multi-Color LLS                   \\ \hline
  \end{tabular}
\end{table}

\begin{table}[hbtp]
  \centering
  \section*{$\blacksquare$ 構成}
  \begin{tabular}{r c l}
    \hline
    1. & \gt{序論}     & \begin{tabular}{l} 二次流れ観測の重要性と\\先行研究(計測手法)について \end{tabular} \\ \hline
    2. & \gt{計測手法} & \begin{tabular}{l} 粒子クラスタを用いた\\粒子追跡アルゴリズム \end{tabular}           \\ \hline
    3. & \gt{数値解析} & \begin{tabular}{l} 数値解析を用いた\\アルゴリズムの性能評価   \end{tabular}           \\ \hline
    4. & \gt{基礎実験} & \begin{tabular}{l} 三角翼後流の計測結果\\ (比較的安定した流れの計測 \end{tabular}    \\ \hline
    5. & \gt{応用実験} & \begin{tabular}{l} 車両モデル後流の計測結果\\ (不安定な流れの計測) \end{tabular}    \\ \hline
    6. & \gt{結論}     &                                                                                       \\ \hline
  \end{tabular}
\end{table}

\section{序論}
\section{計測手法}
\section{数値解析}
\section{基礎実験}
\section{応用実験}
\section{結論}

\section{12月の予定}
\begin{itemize}
  \item 数値解析による性能評価
  \item 車両モデルの計測
  \item 修士論文の執筆
\end{itemize}

\end{document}